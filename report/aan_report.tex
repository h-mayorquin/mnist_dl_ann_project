%\RequirePackage[l2tabu, orthodox]{nag}  %Checks for older packages 

\documentclass[11pt,a4paper]{article}
% \documentclass[10pt]{extreport} $ allos to make the font smaller
\usepackage[utf8]{inputenc}

\usepackage{amsmath}
\usepackage{amsfonts}
\usepackage{indentfirst}
\usepackage{amssymb}
\usepackage{siunitx}  % Units of the metric system  \SI{2.63}{\ohm}   
\usepackage[font={footnotesize}]{caption} %Makes the captions small

\usepackage{algorithm}
\usepackage{algpseudocode}

%% Figures packages
\usepackage[pdftex]{graphicx}
\usepackage{float}   %Es para colocar el objeto flotante exactamente donde se ponga el comando con H
\usepackage{caption}
\usepackage{subcaption}
\graphicspath{{../results/}}
\usepackage{sidecap}  %Para poner figuras a los lados


\usepackage{setspace} % Needed for Pyton syntax highlight
\usepackage{listings}    % Include the listings-package, nice verbatim for code
\usepackage{color}
\usepackage{courier}


\usepackage{cleveref} %puts figure and equation appropiately \cref{} 

\usepackage{natbib} %For bibliography
%\usepackage{cite}
\usepackage{framed} % To use boxes, or frames around text

\usepackage{parskip} %Para saltos entre parrafos
\setlength{\parindent}{0pt} 
\setlength{\parskip}{\baselineskip}
\usepackage[a4paper,margin=0.8in]{geometry}  %%Cambiar los margenes

\newcommand{\HRule}{\rule{\linewidth}{0.5mm}}

%\usepackage{hyperref} %This should be loade after most of the other packages 
% \hypersetup{colorlinks=true}  %Para que los hiperlinks cuando se hagan referencias aparezcan en colores.

% This is for the Python Package
\definecolor{Code}{rgb}{0,0,0}
\definecolor{Decorators}{rgb}{0.5,0.5,0.5}
\definecolor{Numbers}{rgb}{0.5,0,0}
\definecolor{MatchingBrackets}{rgb}{0.25,0.5,0.5}
\definecolor{Keywords}{rgb}{0,0,1}
\definecolor{self}{rgb}{0,0,0}
\definecolor{Strings}{rgb}{0,0.63,0}
\definecolor{Comments}{rgb}{0,0.63,1}
\definecolor{Backquotes}{rgb}{0,0,0}
\definecolor{Classname}{rgb}{0,0,0}
\definecolor{FunctionName}{rgb}{0,0,0}
\definecolor{Operators}{rgb}{0,0,0}
\definecolor{Background}{rgb}{0.98,0.98,0.98}

\lstnewenvironment{python}[1][]{
\lstset{
numbers=left,
numberstyle=\footnotesize,
numbersep=1em,
xleftmargin=1em,
framextopmargin=2em,
framexbottommargin=2em,
showspaces=false,
showtabs=false,
showstringspaces=false,
frame=l,
tabsize=4,
% Basic
basicstyle=\ttfamily\small\setstretch{1},
backgroundcolor=\color{Background},
language=Python,
% Comments
commentstyle=\color{Comments}\slshape,
% Strings
stringstyle=\color{Strings},
morecomment=[s][\color{Strings}]{"""}{"""},
morecomment=[s][\color{Strings}]{'''}{'''},
% keywords
morekeywords={import,from,class,def,for,while,if,is,in,elif,else,not,and,or,print,break,continue,return,True,False,None,access,as,,del,except,exec,finally,global,import,lambda,pass,print,raise,try,assert},
keywordstyle={\color{Keywords}\bfseries},
% additional keywords
morekeywords={[2]@invariant},
keywordstyle={[2]\color{Decorators}\slshape},
emph={self},
emphstyle={\color{self}\slshape},
%
}}{}



\definecolor{dkgreen}{rgb}{0,0.6,0}

\title{Lab12: DD2380 }
\author{
Ramon Heberto Martinez Mayorquin  hramon@kth.se 
Akash Kumar Dhaka  akashd@kth.se 
}



\begin{document}

\begin{titlepage}
\begin{center}
%\includegraphics[width=0.15\textwidth]{logo}\\[1cm]    

\textsc{\LARGE Kungliga Tekniska högskolan}\\[1.0cm]

\textsc{\Large Class: DD2432 Artificial Neural Networks}\\[2.0cm]



\begin{figure}[H]
	\centering
 \includegraphics[width=0.35\textwidth]{Kth_logo.png}
\end{figure}
%\\[1cm]    


% Title
\HRule \\[0.4cm]
{ \huge  Project: Deep Neural Networks in Theano
}\\[0.4cm]
\HRule \\[1.5cm]

% Author and supervisor

Author: Ram\'on Heberto Mart\'inez \\ 
\large Professors: Erik Frans\'en, Pawel Herman  \\ [2.5cm]
%\normalsize Presenta \\
%\large Supervisors 2: Jan Antolik \\[2.5cm]

\textsc{\Large School of Computer Science and Communication }\\ [1.0cm] 
\includegraphics[width=0.15\textwidth]{KTH_black.png}\\[1.5cm] % Controls the distance till the new object 
% Bottom of the page
{\large 31 May of 2015}

\end{center}
\end{titlepage}

%%%%%%%%%%%%%%%%%%%%%%%%%%%%%%%%%%%%
%%%%%%%%%%%%%%%%%%%%%%%%%%%%%%%%%%%%
\section{Introduction}
%%%%%%%%%%%%%%%%%%%%%%%%%%%%%%%%%%%%
%%%%%%%%%%%%%%%%%%%%%%%%%%%%%%%%%%%%

%%%%%%%%%%%%%%%%%%%%%%%%%%%%%%%%%%%%
\subsection{The MNIST Data Set}
The \textbf{MNIST}  data set (\cite{lecun1998mnist}) is a set of images used widely in the field of Machine Learning in general and Visual Recognition in particular. The set consists of hand-written digits from different people. We show a sample of the images in figure \ref{fig:mnist_example} were we can appreciate the variability inter digit and among the digits. 

%% Figure %% 
\begin{center}
\begin{figure}[H]
\centering
\includegraphics[scale=.45]{mnist_example.png} 
\caption{A sample from the MNIST dat set. In total it possess $60000$ training samples and $10000$ images.  We can see that we have both variability within the digits and among the digits..}
\label{fig:mnist_example}
\end{figure} 
\end{center}

The set was formed by combining two data sets of the National Institute of Standars and Technology (NIST) from the United States. One of them consisted on hand 
-written digits by high school students and the other is composed of the same elements but written by employees of the United States Census Bureau. 

%%%%%%%%%%%%%%%%%%%%%%%%%%%%%%%%%%%%
\subsection{Theano}
\textbf{Theano} is a python library that implements symbolic differentiation and evaluation of expressions involving arrays of multiple dimensions (\cite{bergstra2010theano}). The library was developed in the research environment of the University of Montreal and is extensively used in the Deep Learning community as a library that is powerful, efficient and provides ample room for experimentation. 

In this work we will be implementing the algorithms in \textbf{Theano}. Along this work we will mentioned how the capabilities of Theano allow us to carry the task at hand in an easy an convenient way. 

%%%%%%%%%%%%%%%%%%%%%%%%%%%%%%%%%%%%
%%%%%%%%%%%%%%%%%%%%%%%%%%%%%%%%%%%%
\section{Algorithms}
%%%%%%%%%%%%%%%%%%%%%%%%%%%%%%%%%%%%
%%%%%%%%%%%%%%%%%%%%%%%%%%%%%%%%%%%%

In total we implemented three algorithms using the templates that the \textbf{Theano} library provides: the Multilayer Perceptron, a Restricted Boltzmann machine trained by constrastive divergence and a Deep Belief Network. In this section we briefly describe their architecture and how they fit within the architectural constrains of Theano.

%%%%%%%%%%%%%%%%%%%%%%%%%%%%%%%%%%%%
\subsection{Multi Layer Perception}

The Multilayer perceptron is an artificial neural network without recurrent connections (that is feedfoward) that is a generalization of the linear perceptron. Its strength is that it can distinguish data that is not separable. We present the archicture of our network in figure \ref{fig:mlp}. 

\begin{SCfigure}[][h]
  \centering
  \caption{Here we describe the architecture of the Multilayer Perceptron. In the left side of the picture we have the inputs, as inputs the neuron receives the raw data that we are interested in classifying (or using regression), we call the input side the input layer. The input layer should have as many units as dimensions has the data that we are interested in classifying. The layer in between is called the hidden layer and is the responsible or building another representation of our data where the non-linear data is separable. Finally the output layer in the right reveals with its activation to which category the input belongs (or provides the value if it is regression).  }
  \includegraphics[width=0.4\textwidth]%
    {MLP.jpg}% picture filename
	\label{fig:mlp}
\end{SCfigure}


The architecture can be represented mathematically by the following formula

\begin{align*}
f(x) = G(b^2 + W^2 (\underset{Hidden Layer}{s(b^2 + W^1x)}))
\end{align*}

The Hidden layer is built by multiplying the first matrix of coefficients, adding the bias and then a non-linear transformation $s$ to produce the hidden layer as indicated with text in the equation above. After that the Hidden layer is then transformed again by the same process to obtain the input. The superscript in the equation indicates the level of deep that the layer belongs too and is not a power in the mathematical sense. 

Now, let's suppose that we have a data set $D$ composed of inputs $x_i$ and outputs $y_i$. In order to train the algorithm we need to define a \textbf{loss function} that will let us know in which direction we have to move our parameters so our output is better predicted given the model. For the MLP we define the loss function through the likelihood function:

\begin{align*}
\mathcal{L}(\theta = \{W, b\}, D) = \sum_{i=0}^{D} 
\log(P(Y = y^i | x^i , W, b	))
\end{align*}

Where the probability $P(Y = y^i | x^i , W, b	)$ can be defined as a \textbf{softmax function} in the last layer. The prediction consequently is the output layer for which the probability is maximal.  With this in our hands we can converge to our parameters by moving in the direction that minimizes the negative of the likelihood above. The usual way to achieve this is through the \textbf{Backpropagation Algorithm} that allow us to calculate the gradient of the error with respect to all parameters but with the symbolic capabilities of \textbf{Thean} this can be done using only the following lines of code:


\begin{python}
g_W = T.grad(cost=cost, wrt=classifier.W)
g_b = T.grad(cost=cost, wrt=classifier.b)
\end{python}

Where the classifier is just the class instatiation of our algorithm and cost is the negative of the function presented above. After this the update rules is written as a simple as:

\begin{center}
\begin{python}
updates = [(classifier.W, classifier.W - learning_rate * g_W),
           (classifier.b, classifier.b - learning_rate * g_b)]
\end{python}
\end{center}

With both this expressions we can use a function that updates the parameters at every training step and as simple as that Theano performs a step of gradient descent. 

%%%%%%%%%%%%%%%%%%%%%%%%%%%%%%%%%%%%
\subsection{Restricted Boltzmann Machine}

A Restricted Boltzmann Machine is a neural network that has hidden layer and can learn the distribution over the observable layer. They were discovered at the end of the eighties by Paul Smolensky and called Harmonium back then (\cite{mcclelland1986parallel} Chapter 6) but the training algorithms were slow. It was not until the middle of the last decade that fast algorithms to train them were discovered (\cite{hinton2006reducing}). We present a graphical representations of their architecture in the figure \ref{fig:rbm}.


\begin{SCfigure}[][h]
  \centering
  \caption{Here we describe the architecture of the Restricted Boltzmann Machine. put belongs (or provides the value if it is regression).  The visible units are represented to the left and they correspond to the inputs, that is, they are usually the data that we have and we are interesting in modeling. In the right we have the hidden units, this is the part of the system in charge of representing features and hidden relations in our data. In opposition to complete Boltzmann Machine there are no connections within any of the layers, the only connections are between them.  }
  \includegraphics[width=0.35\textwidth]%
    {rbm.png}% picture filename
	\label{fig:rbm}
\end{SCfigure}


The Restricted Boltzman Machine is an energy model. That means that we represent the probability of a particular input in terms of an energy function in the following way:

\begin{align*}
p(v) = \frac{e^{-E(v)}}{Z}
\end{align*}

Where $Z$ is just a normalizing factor that is constructed by summing the denominator in the equation above through all the possible values of the input and makes $p$ a probability distribution. 

The idea here is to make our data the most probably with a certain energy function. In this case though, we have hidden units and we are interesting in the distribution over the observable units. So in order to get that distribution we can marginalize in the usual way:

\begin{align*}
P(v) = \sum_{h} P(v,h) = \sum_{h} \frac{e^{-E(v, h)}}{Z}
\end{align*}

Where the v represents a unit from the visible layer and h a unit from the hidden layer. Being specific in the case of the RBM's the energy function takes the following form:

\begin{align*}
E(v, h) = -bv -ch - hWv 
\end{align*}

Where $b$ and $c$ are the bias term of the visible layer and the hidden layer respectively. Finally $W$ is the matrix that encodes the connections between the layers. Is those parameters the ones that we have to adjust in order to make the system give the highest probability to the actual data that we show to it. 

Following inspiration from Physics we introduce a quantity called the \textbf{Free Energy}: 

\begin{align*}
\mathcal{F}(x) = - \log \sum_{h} e^{E(x, h)}
\end{align*}

Give the energy above the free energy then is:

\begin{align*}
\mathcal{F}(x) = -bv - \sum_i \log_{h_i} e^{h_i(c_i + W_i v)}
\end{align*}

Regarding the question of how to train this type of mode the answer is that we can, as usual, perform gradient descent on the empirical negative log-likelihood which we define as: 


\begin{align*}
\mathcal{L}(\theta, D) = \frac{1}{N} \sum_{x \in D} \log p(x)
\end{align*}

But if we come back to the free energy expression in turns out that we can represent the derivative in the sum above in the following way:

\begin{align*}
- \dfrac{\partial{\log p(x)}}{\partial{\theta}} = \dfrac{\partial{\mathcal{F}(x)}}{\partial{\theta}} - \sum_{y} p(y) \dfrac{\partial{\mathcal{F}(y)}}{\partial{\theta}}
\end{align*}

The term in the left is know as the \textbf{positive face} and ire responsible for increasing the probability of the training data. The second term, on the other hand, is called \textbf{negative phase} and its function is to reduce the probability of the data generated by the model (\cite{yoshua2009learning}).

The positive phase term is easy to compute. We only calculate the gradient of the free energy at our particular example. The second term on the hand requires an average of the gradient of the free energy over the total number of possible configurations which is in most of the cases not tractable. In order to simplify the task what we do in to use \textbf{Monte Carlo} methods. In particular we create a Markov Chain and perform Gibbs sampling to obtain samples from the underlying distribution. 

An important further step to make the training of the RBM's fast enough is the form in which we computer the Gibbs sampling. Two famous tricks are \textbf{Constrastive Divergence} (CD-k) where the first term of the chain is sampled from the training data and \textbf{Persistent CD} where the state of the chain is preserved at the end of the sampling for the next iteration. 

It is important to mention that usually we need to express the free energy as an analytically expression but with the symbolic capabilities of Theano the code allows us to actually calculate the gradient from the expression above. 

\begin{python}
   # determine gradients on RBM parameters
        # note that we only need the sample at the end of the chain
        chain_end = nv_samples[-1]

        cost = T.mean(self.free_energy(self.input)) - T.mean(
            self.free_energy(chain_end))
        # We must not compute the gradient through the gibbs sampling
        gparams = T.grad(cost, self.params, consider_constant=[chain_end])
\end{python}

Where the variable chain end was obtained from the Gibbs sampling mechanism (is the last sample of the visible layer). The mean over there is because we are using a min-batches approach to regression and therefore we move in the direction of the batch. Note that this symbolical calculation of the gradient saves us a lot of problems in the domain of numerical stability and makes the algorithm overalls cleared (although harder to implement because it requires a different type of thinking).

%%%%%%%%%%%%%%%%%%%%%%%%%%%%%%%%%%%%
\subsection{Deep Belief Networks}




%%%%%%%%%%%%%%%%%%%%%%%%%%%%%%%%%%%%
\section{Results}
%%%%%%%%%%%%%%%%%%%%%%%%%%%%%%%%%%%%

%%%%%%%%%%%%%%%%%%%%%%%%%%%%%%%%%%%%
\section{Discussion}
%%%%%%%%%%%%%%%%%%%%%%%%%%%%%%%%%%%%

\bibliographystyle{plainnat}
\bibliography{ref_deep_learning}
\end{document}